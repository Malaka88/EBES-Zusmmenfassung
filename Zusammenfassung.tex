% Options for packages loaded elsewhere
\PassOptionsToPackage{unicode}{hyperref}
\PassOptionsToPackage{hyphens}{url}
%
\documentclass[
]{article}
\usepackage{amsmath,amssymb}
\usepackage{lmodern}
\usepackage{iftex}
\ifPDFTeX
  \usepackage[T1]{fontenc}
  \usepackage[utf8]{inputenc}
  \usepackage{textcomp} % provide euro and other symbols
\else % if luatex or xetex
  \usepackage{unicode-math}
  \defaultfontfeatures{Scale=MatchLowercase}
  \defaultfontfeatures[\rmfamily]{Ligatures=TeX,Scale=1}
\fi
% Use upquote if available, for straight quotes in verbatim environments
\IfFileExists{upquote.sty}{\usepackage{upquote}}{}
\IfFileExists{microtype.sty}{% use microtype if available
  \usepackage[]{microtype}
  \UseMicrotypeSet[protrusion]{basicmath} % disable protrusion for tt fonts
}{}
\makeatletter
\@ifundefined{KOMAClassName}{% if non-KOMA class
  \IfFileExists{parskip.sty}{%
    \usepackage{parskip}
  }{% else
    \setlength{\parindent}{0pt}
    \setlength{\parskip}{6pt plus 2pt minus 1pt}}
}{% if KOMA class
  \KOMAoptions{parskip=half}}
\makeatother
\usepackage{xcolor}
\IfFileExists{xurl.sty}{\usepackage{xurl}}{} % add URL line breaks if available
\IfFileExists{bookmark.sty}{\usepackage{bookmark}}{\usepackage{hyperref}}
\hypersetup{
  hidelinks,
  pdfcreator={LaTeX via pandoc}}
\urlstyle{same} % disable monospaced font for URLs
\setlength{\emergencystretch}{3em} % prevent overfull lines
\providecommand{\tightlist}{%
  \setlength{\itemsep}{0pt}\setlength{\parskip}{0pt}}
\setcounter{secnumdepth}{-\maxdimen} % remove section numbering
\ifLuaTeX
  \usepackage{selnolig}  % disable illegal ligatures
\fi

\author{}
\date{}

\begin{document}

\hypertarget{osi}{%
\section{OSI}\label{osi}}

\begin{table}[]
\begin{tabular}{lllllll}
                               & Daten                                          & Layer                     & Name                                       & Protokolbeispiel                                                    & Hardware                               &                                                                \\
                               & \cellcolor[HTML]{9AFF99}                       & \cellcolor[HTML]{9AFF99}7 & \cellcolor[HTML]{9AFF99}Application Layer  & \cellcolor[HTML]{9AFF99}HTTP, FTP, DNS, SNMP, Telnet                & \cellcolor[HTML]{9AFF99}PC             & \cellcolor[HTML]{9AFF99}Netzwerkprozess zur Applikation        \\
                               & \cellcolor[HTML]{9AFF99}                       & \cellcolor[HTML]{9AFF99}6 & \cellcolor[HTML]{9AFF99}Presentation Layer & \cellcolor[HTML]{9AFF99}SSL, TLS                                    & \cellcolor[HTML]{9AFF99}               & \cellcolor[HTML]{9AFF99}Verschlüsselung, Darstellung der Daten \\
                               & \multirow{-3}{*}{\cellcolor[HTML]{9AFF99}Data} & \cellcolor[HTML]{9AFF99}5 & \cellcolor[HTML]{9AFF99}Session Layer      & \cellcolor[HTML]{9AFF99}NetBIOS, PPTP                               & \cellcolor[HTML]{9AFF99}               & \cellcolor[HTML]{9AFF99}                                       \\
\multirow{-4}{*}{Host Layers}  & \cellcolor[HTML]{67FD9A}Segments               & \cellcolor[HTML]{67FD9A}4 & \cellcolor[HTML]{67FD9A}Transport Layer    & \cellcolor[HTML]{67FD9A}TCP, UDP                                    & \cellcolor[HTML]{67FD9A}               & \cellcolor[HTML]{67FD9A}Ende zu Ende verbindung (Handshake)    \\
                               & \cellcolor[HTML]{FFFC9E}Packets                & \cellcolor[HTML]{FFFC9E}3 & \cellcolor[HTML]{FFFC9E}Network Layer      & \cellcolor[HTML]{FFFC9E}IP, ARP, ICMP, IPSec                        & \cellcolor[HTML]{FFFC9E}Switch, Router & \cellcolor[HTML]{FFFC9E}Logische Adressierung und Pfadfindung  \\
                               & \cellcolor[HTML]{FFCE93}Frames                 & \cellcolor[HTML]{FFCE93}2 & \cellcolor[HTML]{FFCE93}Data Link Layer    & \cellcolor[HTML]{FFCE93}PPP, ATM, Ethernet                          & \cellcolor[HTML]{FFCE93}Switch, Bridge & \cellcolor[HTML]{FFCE93}Physikalische Adressierung mittels MAC \\
\multirow{-3}{*}{Media Layers} & \cellcolor[HTML]{FFCCC9}Bits                   & \cellcolor[HTML]{FFCCC9}1 & \cellcolor[HTML]{FFCCC9}Physical Layer     & \cellcolor[HTML]{FFCCC9}Ethernet, USB, Bluetooth, IEEE802.11 (Wifi) & \cellcolor[HTML]{FFCCC9}Kabel, Hubs    & \cellcolor[HTML]{FFCCC9}Media, signal und binäre Übertragung  
\end{tabular}
\end{table}

\hypertarget{embedded-systems}{%
\section{Embedded Systems}\label{embedded-systems}}

\begin{quote}
Ein \textbf{verteiltes System} besteht aus Komponenten, die räumlich
oder logisch verteilt sind und mittels einer Kopplung bzw. Vernetzung
zum Erreichen der Funktionalität des Gesamtsystems beitragen.
\end{quote}

\begin{quote}
Ein \textbf{Steuergerät} (engl. Electronic Control Unit, ECU) ist die
physikalische Umsetzung eines eingebetteten Systems. In mechatronischen
Systemen bilden Steuergeräte und Sensorik/Aktuatorik oft eine Einheit.
\end{quote}

\begin{quote}
Wird Elektronik zur Steuerung und Regelung mechanischer Vorgänge
räumlich eng mit den mechanischen Systembestandteilen verbunden, so
sprechen wir von einem \textbf{mechatronischen System}. Der
Forschungsbereich der sich mit der Entwicklung mechatronischer Systeme
befasst nennt sich \textbf{Mechatronik}.
\end{quote}

\hypertarget{klassifikation-und-charakteristika}{%
\subsection{Klassifikation und
Charakteristika}\label{klassifikation-und-charakteristika}}

\hypertarget{technische-auspregung}{%
\subsubsection{Technische Auspregung}\label{technische-auspregung}}

\begin{itemize}
\tightlist
\item
  \textbf{kontinuierliche} Systeme
\item
  \textbf{diskreten} Systeme
\item
  \textbf{verteilten} Systeme
\item
  \textbf{monolithischen} Systeme
\item
  \textbf{hybriden} Systeme (sowohl kontinuierlichs als auch diskretes
  Verhalten)
\end{itemize}

\begin{quote}
Systeme die sowohl kontinuierliche (analoge), als auch diskrete
Datenteile (wertkontinuierlich) verarbeiten und/oder sowohl über
kontinuierliche Zeiträume (zeitkontinuierlich), als auch zu diskreten
Zeitbpunkten mit ihrer Umgebung interagieren, heißen \textbf{hybrid
Systeme}.
\end{quote}

\hypertarget{sicherheitsrelevanz}{%
\subsubsection{Sicherheitsrelevanz}\label{sicherheitsrelevanz}}

\begin{itemize}
\tightlist
\item
  \textbf{Sicherheitskritische} Systeme (wenn Menschenleben oder die
  Unversehrtheit von Einrichtungen abhängt) z.B. Avionik, Medizintechnik
  und Kraftfahrzeugbereich
\item
  \textbf{nicht Sicherheitskritische} Systeme z.B. Konsumelektronik
  hauptsächlich nicht Sicherheitskritische Systeme
\item
  \textbf{Zeitkritische} Systeme
\item
  \textbf{nicht Zeitkritische} Systeme
\end{itemize}

\hypertarget{produktkategorien}{%
\subsubsection{Produktkategorien}\label{produktkategorien}}

\begin{itemize}
\tightlist
\item
  Telekommunikation ( Telefon, Fax, etc.)
\item
  Haushalt (Waschmaschine, Mikrowelle, Fernseher, etc.)
\item
  Periphere Geräte (Tastatur, Modem, Drucker, etc.)
\item
  Bürotechnik (Kopierer, Schreibmaschine etc.)
\item
  Geräte für Freizeit, Hobby und Garten
\item
  Automobiltechnik (ABS, Wegfahrsperre, Navigationssysteme, etc.)

  \begin{itemize}
  \tightlist
  \item
    Massenmarkt für eingebettete Systeme
  \item
    häufig Zeit- und zunehmend auch sicherheitskritisch
  \end{itemize}
\item
  Öffentlicher Verkehr (Fahrkartenautomat, etc.)
\item
  Luft- und Raumfahrttechnik
\item
  Fertigungstechnik
\item
  Steuerungs- und Regelungstechnik, Medizintechnik, Umwelttechnik,
  Militärtechnik
\item
  Avionik (Flugzeugbau)
\item
  uvm
\end{itemize}

\hypertarget{klassifikation}{%
\subsubsection{Klassifikation}\label{klassifikation}}

\hypertarget{transformationelle-systeme}{%
\paragraph{Transformationelle
Systeme}\label{transformationelle-systeme}}

\begin{itemize}
\tightlist
\item
  Eingaben müssen zu beginn der Systemverarbeitung vollständig vorliegen
\item
  Ausgaben nur verfügbar wenn die Verarbeitung vollständig terminiert
\item
  Keine interaktion während der Verarbeitung möglich (=\textgreater{}
  kein Einfluss auf die Ergebnisse)
\end{itemize}

\hypertarget{interaktive-systeme}{%
\paragraph{Interaktive Systeme}\label{interaktive-systeme}}

\begin{itemize}
\tightlist
\item
  Ausgabe nicht nur bei Terminierung
\item
  Interaktion und Synchronisierung mit der Umgebung
\item
  Interaktion wird durch das System bestimmt
\item
  proaktive Synchronisierung durch das System
\end{itemize}

\hypertarget{reaktive-systeme}{%
\paragraph{reaktive Systeme}\label{reaktive-systeme}}

\begin{quote}
Ein reaktives System kann aus Software und/oder Hardware bestehen und
setzt Eingabeereignisse deren zeitliches Auftreten meist nicht
vorhergesagt werden kann - oftmals aber nicht notwendigerweise unter
EInhaltung von Zeitvorgaben - in Ausgabeereignisse um.
\end{quote}

\begin{itemize}
\tightlist
\item
  Synchronisierung dich Systemumgebung
\item
  Reagieren auf ihre Umwelt
\item
  Arbeiten häufig Nebenläufig
\item
  müssen sehr zuverlässig sein
\item
  müssen Zeitschranken einhalten (Echtzeitsysteme)
\item
  sind in Hardware als auch Software realisiert
\item
  werden in komplexen, verteilten Systemplattformen implementiert
\item
  Die funktionale Korrektheit ist ein wichtiger Faktor der Entwicklung
\item
  eingebettet in komplexe bsp. mechanische, chemische oder biologische
  Systemumgebung =\textgreater{} eingebettete Systeme
\end{itemize}

\hypertarget{bestandteile}{%
\subsection{Bestandteile}\label{bestandteile}}

\begin{itemize}
\tightlist
\item
  Kontrolleinheit
\item
  Regelstrecke
\item
  Benutzerschnittstelle
\item
  technische Systemumgebung
\item
  menschliche Systembenutzer
\end{itemize}

\hypertarget{definitionen}{%
\section{Definitionen}\label{definitionen}}

\begin{quote}
Als \textbf{Steuergerät} (engl. \textbf{Electronic Control Unit},
\textbf{ECU}) wird die eigentliche Steuereinheit eines mechatronischen
Systems verstanden.

Steuergeräte sind im Prinzip wie folgt aufgebaut: Die Kernkomponente des
Steuergeräts stellt ein Mikrocontroller oder Mikroprozessor (Beispiele:
Power PC, Alpha, PC) dar. Zusätzlich kann es optional ein externes RAM
und/oder ROM besitzen sowie sonstige Peripherie und Bauelemente.

Mikrocontroller kennzeichnen eine Klasse von Mikroprozessoren, die auf
den speziellen Anwendungsbereich der Steuerung von Prozessen
zugeschnitten sind. Wir betrachten im Folgenden einige Spezialfälle
genauer.
\end{quote}

\begin{quote}
Ein \textbf{ASIP} (\textbf{applikationsspezifischer Prozessor}) ist ein
Prozessor, der von seiner Struktur als auch von seinem Befehlssatz her
auf seinen Einsatz für bestimmte Anwendungen hin optimiert ist. Er
besitzt spezielle Instruktionssätze, funktionale Einheiten, Register und
spezielle Verbindungsttrukturen.

\begin{itemize}
\tightlist
\item
  kostengünstiger aufgrund abgespeckter Prozessoren
\item
  aufgrund Programmierbarkeit flexibel
\item
  höhere Verarbeitungsgeschwindigkeit und geringere Leistungsaufnahme
  aufgrund optimierter Strukturen
\item
  \textbf{Nachteil} = komplizierte und aufwendige Entwicklung
\end{itemize}
\end{quote}

\begin{quote}
Ein \textbf{DSP} ist ein spezieller Mikroprozessor, der Befehle (und
damit Verarbeitungseinheiten) zur Durchführung von
Signal-Verarbeitungs-Aufgaben (bsp. Fast Fourier Transformation,
\textbf{FFT}) besitzt.

\begin{itemize}
\tightlist
\item
  Optimiert auf die häufig vorkommenden Opperationen bei der digitalen
  Signalverarbeitung (bsp. schnelle Multiplikation)
\item
  hoher Stellenwert ist die Effizienz dieser Operationen
\item
  \textbf{Einsatz} bsp. MP3 decoder oder Sprachsignalverarbeitung oder
  Bildverarbeitung
\end{itemize}
\end{quote}

\begin{quote}
Das \textbf{Field Programmable Gate Array} (\textbf{FPGA}) ist ein
komplexer, programmierbarer Logikbaustein, der zum Aufbau digitaler,
logischer Schaltungen dient. Er besteht im Wesentlichen aus einzelnen
Funktionsblöcken, die in einer regelmäßigen Struktur (\textbf{Array})
angeordnet sind, und einen Netzwerk von Verbindungen zwischen diesen
Blöcken. Bei FPGAs wird die Implementierung von logischen Funktionen
hauptsächlich durch die Programmierung der Verbindungsleitungen zwischen
den Logikblöcken erreicht.

\begin{itemize}
\tightlist
\item
  zwei Arten von FPGAs

  \begin{itemize}
  \tightlist
  \item
    rekonfigurierbare (unter verwendung von Speichertechnologien wie
    SRAM)

    \begin{itemize}
    \tightlist
    \item
      Nachteil: sie sind flüchtig
    \end{itemize}
  \item
    nicht rekonfigurierbare (einmal Konfiguriert immer Konfiguriert,
    umsetzung durch physikalische Zerstörung der nicht benötigten
    Verbindungsleitungen)
  \end{itemize}
\item
  Realisierung von Speicherzellen möglich
\item
  Eignung zur Realisierung von Steuererwerken (in Form endlicher
  Automaten)
\item
  Im Gegensatz zu gewöhnlichen \textbf{Gate Arrays} (\textbf{GA}) sind
  FPGAs programmierbare Logikbausteine, deren Funktionalität durch das
  Zusammenschalten verschiedener Funktionsblöcke erreicht wird
\end{itemize}
\end{quote}

\end{document}
